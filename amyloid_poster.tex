\documentclass[final]{beamer}\usepackage[]{graphicx}\usepackage[]{color}
%% maxwidth is the original width if it is less than linewidth
%% otherwise use linewidth (to make sure the graphics do not exceed the margin)
\makeatletter
\def\maxwidth{ %
  \ifdim\Gin@nat@width>\linewidth
    \linewidth
  \else
    \Gin@nat@width
  \fi
}
\makeatother

\definecolor{fgcolor}{rgb}{0.345, 0.345, 0.345}
\newcommand{\hlnum}[1]{\textcolor[rgb]{0.686,0.059,0.569}{#1}}%
\newcommand{\hlstr}[1]{\textcolor[rgb]{0.192,0.494,0.8}{#1}}%
\newcommand{\hlcom}[1]{\textcolor[rgb]{0.678,0.584,0.686}{\textit{#1}}}%
\newcommand{\hlopt}[1]{\textcolor[rgb]{0,0,0}{#1}}%
\newcommand{\hlstd}[1]{\textcolor[rgb]{0.345,0.345,0.345}{#1}}%
\newcommand{\hlkwa}[1]{\textcolor[rgb]{0.161,0.373,0.58}{\textbf{#1}}}%
\newcommand{\hlkwb}[1]{\textcolor[rgb]{0.69,0.353,0.396}{#1}}%
\newcommand{\hlkwc}[1]{\textcolor[rgb]{0.333,0.667,0.333}{#1}}%
\newcommand{\hlkwd}[1]{\textcolor[rgb]{0.737,0.353,0.396}{\textbf{#1}}}%

\usepackage{framed}
\makeatletter
\newenvironment{kframe}{%
 \def\at@end@of@kframe{}%
 \ifinner\ifhmode%
  \def\at@end@of@kframe{\end{minipage}}%
  \begin{minipage}{\columnwidth}%
 \fi\fi%
 \def\FrameCommand##1{\hskip\@totalleftmargin \hskip-\fboxsep
 \colorbox{shadecolor}{##1}\hskip-\fboxsep
     % There is no \\@totalrightmargin, so:
     \hskip-\linewidth \hskip-\@totalleftmargin \hskip\columnwidth}%
 \MakeFramed {\advance\hsize-\width
   \@totalleftmargin\z@ \linewidth\hsize
   \@setminipage}}%
 {\par\unskip\endMakeFramed%
 \at@end@of@kframe}
\makeatother

\definecolor{shadecolor}{rgb}{.97, .97, .97}
\definecolor{messagecolor}{rgb}{0, 0, 0}
\definecolor{warningcolor}{rgb}{1, 0, 1}
\definecolor{errorcolor}{rgb}{1, 0, 0}
\newenvironment{knitrout}{}{} % an empty environment to be redefined in TeX

\usepackage{alltt}
\usepackage{grffile}
\mode<presentation>{\usetheme{CambridgeUSPOL}}

\usepackage[utf8]{inputenc}
\usepackage{amsfonts}
\usepackage{amsmath}
\usepackage{natbib}
\usepackage{graphicx}
\usepackage{array,booktabs,tabularx}
\newcolumntype{Z}{>{\centering\arraybackslash}X}

% rysunki
\usepackage{tikz}
\usepackage{ifthen}
\usepackage{xxcolor}
\usetikzlibrary{arrows}
\usetikzlibrary[topaths]
\usetikzlibrary{decorations.pathreplacing}
%\usepackage{times}\usefonttheme{professionalfonts}  % times is obsolete
\usefonttheme[onlymath]{serif}
\boldmath
\usepackage[orientation=portrait,size=a0,scale=1.4,debug]{beamerposter}                       % e.g. for DIN-A0 poster
%\usepackage[orientation=portrait,size=a1,scale=1.4,grid,debug]{beamerposter}                  % e.g. for DIN-A1 poster, with optional grid and debug output
%\usepackage[size=custom,width=200,height=120,scale=2,debug]{beamerposter}                     % e.g. for custom size poster
%\usepackage[orientation=portrait,size=a0,scale=1.0,printer=rwth-glossy-uv.df]{beamerposter}   % e.g. for DIN-A0 poster with rwth-glossy-uv printer check
% ...
%

\usecolortheme{seagull}
\useinnertheme{rectangles}
\setbeamercolor{item projected}{bg=darkred}
% \setbeamertemplate{enumerate items}[default]
\setbeamertemplate{navigation symbols}{}
\setbeamercovered{transparent}
\setbeamercolor{block title}{fg=darkred}
\setbeamercolor{local structure}{fg=darkred}

\setbeamercolor*{enumerate item}{fg=darkred}
\setbeamercolor*{enumerate subitem}{fg=darkred}
\setbeamercolor*{enumerate subsubitem}{fg=darkred}

\setbeamercolor*{itemize item}{fg=darkred}
\setbeamercolor*{itemize subitem}{fg=darkred}
\setbeamercolor*{itemize subsubitem}{fg=darkred}

\newlength{\columnheight}
\setlength{\columnheight}{95cm}
\renewcommand{\thetable}{}
\def\andname{,}
\authornote{}

\renewcommand{\APACrefatitle}[2]{}
\renewcommand{\bibliographytypesize}{\footnotesize} 
\renewcommand{\APACrefYearMonthDay}[3]{%
  {\BBOP}{#1}
  {\BBCP}
}
\IfFileExists{upquote.sty}{\usepackage{upquote}}{}
\begin{document}






\date{}
\author{Micha\l{}  Burdukiewicz\inst{1}*, Piotr Sobczyk\inst{2}, Pawe\l{} Mackiewicz\inst{1}, Stefan R\"odiger\inst{3} \\
*michalburdukiewicz@gmail.com}

\institute{\small{\textsuperscript{1}University of Wroc\l{}aw, Department of Genomics

\vspace{0.2cm}

\textsuperscript{2}Wroc\l{}aw University of Technology, Faculty of Pure and Applied Mathematics

\vspace{0.2cm}

\textsuperscript{3}Brandenburg University of Technology Cottbus-Senftenberg, Institute of Biotechnology}

}
\title{\huge dpcR: web server and R package for analysis of digital PCR experiments}

\begin{frame}
  \begin{columns}
    \begin{column}{.46\textwidth}
      \begin{beamercolorbox}[center,wd=\textwidth]{postercolumn}
        \begin{minipage}[T]{.95\textwidth}
          \parbox[t][\columnheight]{\textwidth}
            {
    
        
    \begin{block}{Introduction}
      Secretory signal peptides:
        \begin{itemize}
          \item are short (20-30 residues) N-terminal amino acid sequences,
            \item direct a protein to the endomembrane system and next to the extracellular localization,
            \item possess three distinct domains with variable length and characteristic amino acid composition~\citep{hegde_surprising_2006}.
            \item are universal enough to direct properly proteins in different secretory systems; artifically introduced bacterial signal peptides can guide proteins in mammals~\citep{nagano_2014} and plants~\citep{moeller_2009},
            \item tag among others hormons, immune system proteins, structural proteins, and metabolic enzymes.
        \end{itemize}
    \end{block}
    
    \vfill
    
    \begin{block}{Organization of signal peptide}

      \begin{itemize}
        \item n-region: mostly basic residues~\citep{nielsen_prediction_1998},
        \item h-region: strongly hydrophobic residues~\citep{nielsen_prediction_1998},
        \item c-region: a few polar, uncharged residues~\citep{jain_signal_1994}.
      \end{itemize}
    \end{block}
    \vfill
    
    
    \begin{block}{Hidden semi-Markov model (HSMM) of a signal peptide}
      Assumptions of the model:
      \begin{itemize}
        \item the observable distribution of amino acids arises due to being in a certain region (state),
        \item a duration of the state (the length of given region) is modeled by a probability distribution (other than geometric distribution as in typical hidden Markov models).
      \end{itemize}
    \end{block}
    \vfill
            }
        \end{minipage}
      \end{beamercolorbox}
    \end{column}
    
    
%new column ------------------------------------------------------    
    
    \begin{column}{.54\textwidth}
      \begin{beamercolorbox}[center,wd=\textwidth]{postercolumn}
        \begin{minipage}[T]{.95\textwidth}  
          \parbox[t][\columnheight]{\textwidth}
            {
     
    \begin{block}{Conclusions}
      Hidden semi-Markov models can be used to accurately predict the presence of secretory signal peptides effectively extracting information from very data sets. Prediction of cleavage site position still requires refinement.
    \end{block}
    \vfill 
    
        \begin{block}{Availability and funding}
        \footnotesize{
      signal.hsmm web server: 
      
      \url{www.smorfland.uni.wroc.pl/signalhsmm}
        }
        
        This research was partially funded by KNOW Consortium.
    \end{block}
    \vfill 
     
     
    \begin{block}{Bibliography}
    \tiny{
      \bibliographystyle{apalike}
      \bibliography{amyloids}
    }
    \end{block}
    \vfill
            }
        \end{minipage}
      \end{beamercolorbox}
    \end{column}
  \end{columns}  
\end{frame}
\end{document}
