\documentclass[final]{beamer}\usepackage[]{graphicx}\usepackage[]{color}
%% maxwidth is the original width if it is less than linewidth
%% otherwise use linewidth (to make sure the graphics do not exceed the margin)
\makeatletter
\def\maxwidth{ %
  \ifdim\Gin@nat@width>\linewidth
    \linewidth
  \else
    \Gin@nat@width
  \fi
}
\makeatother

\definecolor{fgcolor}{rgb}{0.345, 0.345, 0.345}
\newcommand{\hlnum}[1]{\textcolor[rgb]{0.686,0.059,0.569}{#1}}%
\newcommand{\hlstr}[1]{\textcolor[rgb]{0.192,0.494,0.8}{#1}}%
\newcommand{\hlcom}[1]{\textcolor[rgb]{0.678,0.584,0.686}{\textit{#1}}}%
\newcommand{\hlopt}[1]{\textcolor[rgb]{0,0,0}{#1}}%
\newcommand{\hlstd}[1]{\textcolor[rgb]{0.345,0.345,0.345}{#1}}%
\newcommand{\hlkwa}[1]{\textcolor[rgb]{0.161,0.373,0.58}{\textbf{#1}}}%
\newcommand{\hlkwb}[1]{\textcolor[rgb]{0.69,0.353,0.396}{#1}}%
\newcommand{\hlkwc}[1]{\textcolor[rgb]{0.333,0.667,0.333}{#1}}%
\newcommand{\hlkwd}[1]{\textcolor[rgb]{0.737,0.353,0.396}{\textbf{#1}}}%

\usepackage{framed}
\makeatletter
\newenvironment{kframe}{%
 \def\at@end@of@kframe{}%
 \ifinner\ifhmode%
  \def\at@end@of@kframe{\end{minipage}}%
  \begin{minipage}{\columnwidth}%
 \fi\fi%
 \def\FrameCommand##1{\hskip\@totalleftmargin \hskip-\fboxsep
 \colorbox{shadecolor}{##1}\hskip-\fboxsep
     % There is no \\@totalrightmargin, so:
     \hskip-\linewidth \hskip-\@totalleftmargin \hskip\columnwidth}%
 \MakeFramed {\advance\hsize-\width
   \@totalleftmargin\z@ \linewidth\hsize
   \@setminipage}}%
 {\par\unskip\endMakeFramed%
 \at@end@of@kframe}
\makeatother

\definecolor{shadecolor}{rgb}{.97, .97, .97}
\definecolor{messagecolor}{rgb}{0, 0, 0}
\definecolor{warningcolor}{rgb}{1, 0, 1}
\definecolor{errorcolor}{rgb}{1, 0, 0}
\newenvironment{knitrout}{}{} % an empty environment to be redefined in TeX

\usepackage{alltt}
\usepackage{grffile}
\mode<presentation>{\usetheme{CambridgeUSPOL}}

\usepackage[utf8]{inputenc}
\usepackage{amsfonts}
\usepackage{amsmath}
\usepackage{natbib}
\usepackage{graphicx}
\usepackage{array,booktabs,tabularx}
\usepackage{colortbl, xcolor}
\newcolumntype{Z}{>{\centering\arraybackslash}X}

% rysunki
\usepackage{tikz}
\usepackage{ifthen}
\usepackage{xxcolor}
\usetikzlibrary{arrows}
\usetikzlibrary[topaths]
\usetikzlibrary{decorations.pathreplacing}
%\usepackage{times}\usefonttheme{professionalfonts}  % times is obsolete
\usefonttheme[onlymath]{serif}
\boldmath
\usepackage[orientation=portrait,size=a0,scale=1.4,debug]{beamerposter}                       % e.g. for DIN-A0 poster
%\usepackage[orientation=portrait,size=a1,scale=1.4,grid,debug]{beamerposter}                  % e.g. for DIN-A1 poster, with optional grid and debug output
%\usepackage[size=custom,width=200,height=120,scale=2,debug]{beamerposter}                     % e.g. for custom size poster
%\usepackage[orientation=portrait,size=a0,scale=1.0,printer=rwth-glossy-uv.df]{beamerposter}   % e.g. for DIN-A0 poster with rwth-glossy-uv printer check
% ...
%

\usecolortheme{seagull}
\useinnertheme{rectangles}
\setbeamercolor{item projected}{bg=darkred}
% \setbeamertemplate{enumerate items}[default]
\setbeamertemplate{caption}{\insertcaption} 
\setbeamertemplate{navigation symbols}{}
\setbeamercovered{transparent}
\setbeamercolor{block title}{fg=darkred}
\setbeamercolor{local structure}{fg=darkred}

\setbeamercolor*{enumerate item}{fg=darkred}
\setbeamercolor*{enumerate subitem}{fg=darkred}
\setbeamercolor*{enumerate subsubitem}{fg=darkred}

\setbeamercolor*{itemize item}{fg=darkred}
\setbeamercolor*{itemize subitem}{fg=darkred}
\setbeamercolor*{itemize subsubitem}{fg=darkred}

\newlength{\columnheight}
\setlength{\columnheight}{96cm}
\renewcommand{\thetable}{}
\def\andname{,}
\authornote{}

\renewcommand{\APACrefatitle}[2]{}
\renewcommand{\bibliographytypesize}{\footnotesize} 
\renewcommand{\APACrefYearMonthDay}[3]{%
  {\BBOP}{#1}
  {\BBCP}
}
\IfFileExists{upquote.sty}{\usepackage{upquote}}{}
\begin{document}




\date{}
\author{Micha\l{} Burdukiewicz\inst{1}, Piotr Sobczyk\inst{2}, Pawe\l{} Mackiewicz\inst{1} and Ma\l{}gorzata Kotulska\inst{3}\\
\small{*michalburdukiewicz@gmail.com}}


\institute{\small{\textsuperscript{1}University of Wroc\l{}aw, Department of Genomics 

\textsuperscript{2}Wroc\l{}aw University of Technology, Faculty of Pure and Applied Mathematics

\textsuperscript{3}Wroc\l{}aw University of Technology, Department of Biomedical Engineering}
}
}
\title{\huge biogram: a toolkit for biological n-gram analysis}

\begin{frame}
\begin{columns}
\begin{column}{.485\textwidth}
\begin{beamercolorbox}[center,wd=\textwidth]{postercolumn}
\begin{minipage}[T]{.95\textwidth}
\parbox[t][\columnheight]{\textwidth}
{
\begin{block}{Introduction}
N-grams (k-tuples) are vectors of n characters derived from input sequence(s). They may form continuous sub-sequences or be discontinuous. 
Important n-gram parameter is its position. Instead of just counting n-grams, one may want to count how many n-grams occur at a given position in multiple (e.g. related) sequences.

Originally developed for natural language processing, n-grams are also used in genomics~\citep{fang2011}, transcriptomics~\citep{wang2014} and proteomics~\citep{guo2014}.

\small{
       \begin{columns}[c] % the "c" option specifies center vertical alignment
    \column{.5\textwidth} 
% latex table generated in R 3.2.5 by xtable 1.8-2 package
% Sat May 28 02:16:46 2016
\begin{table}[ht]
\centering
\begin{tabular}{rllllll}
  \hline
 & P1 & P2 & P3 & P4 & P5 & P6 \\ 
  \hline
S1 & G & C & C & T & A & A \\ 
  S2 & T & G & C & G & G & A \\ 
  S3 & T & T & G & T & C & G \\ 
   \hline
\end{tabular}
\caption{Sample sequences.  S - sequence, P - postion.} 
\end{table}

      
      
     % column designated by a command

    \column{.5\textwidth}
    
% latex table generated in R 3.2.5 by xtable 1.8-2 package
% Sat May 28 02:16:46 2016
\begin{table}[ht]
\centering
\begin{tabular}{rrrrr}
  \hline
 & A & C & G & T \\ 
  \hline
S1 & 2 & 2 & 1 & 1 \\ 
  S2 & 1 & 1 & 3 & 1 \\ 
  S3 & 0 & 1 & 2 & 3 \\ 
   \hline
\end{tabular}
\caption{Unigram counts.} 
\end{table}


    \end{columns}

% latex table generated in R 3.2.5 by xtable 1.8-2 package
% Sat May 28 02:16:46 2016
\begin{table}[ht]
\centering
\begin{tabular}{rrrrrrrrrrrrrr}
  \hline
 & P1\_A & P2\_A & P3\_A & P4\_A & P5\_A & P6\_A & P1\_C & P2\_C & P3\_C & P4\_C & P5\_C & P6\_C & P1\_G \\ 
  \hline
S1 & 0 & 0 & 0 & 0 & 1 & 1 & 0 & 1 & 1 & 0 & 0 & 0 & 1 \\ 
  S2 & 0 & 0 & 0 & 0 & 0 & 1 & 0 & 0 & 1 & 0 & 0 & 0 & 0 \\ 
  S3 & 0 & 0 & 0 & 0 & 0 & 0 & 0 & 0 & 0 & 0 & 1 & 0 & 0 \\ 
   \hline
\end{tabular}
\caption{A fraction of possible unigrams with position information.} 
\end{table}

}    
    \end{block}


\begin{block}{Curse of dimensionality}
    
Even when we limit ourselves to only continuous positioned n-grams build on $m$ possible
characters, feature space growths rapidly with the number of elements in n-gram
($n$) and the length of the sequence ($L$).    
    
The number of possible positioned n-grams: 

\begin{center}
\scalebox{0.85}{
$
n_{\text{max}} = L \times m^n
$
}

\\


\scalebox{0.91}{  
\begin{knitrout}
\definecolor{shadecolor}{rgb}{0.969, 0.969, 0.969}\color{fgcolor}

{\centering \includegraphics[width=\maxwidth]{figure/unnamed-chunk-4-1} 

}



\end{knitrout}
}
\end{center}
    \end{block}
    \vfill
    
    
    \begin{block}{Feature selecting permutation tests}
    Model and statistic independent permutation tests can be used to filter features obtained through counting n-grams.
    
    During a permutation test class labels are randomly exchanged during computation of a significance statistic. p-values are defined as:
    
\begin{center}
\scalebox{0.85}{
$      
\textnormal{p-value} = \frac{N_{T_P > T_R}}{N}
$
}
\end{center}

where $N_{T_P > T_R}$ is number of times when $T_P$ (permuted test statistic) was more extreme than $T_R$ (test statistic for non-permuted data).

Permutation tests are computationally expensive (especially considering precise estimation of small p-values, because the number of permutations is inversely proportional to the interval between p-values).
      
    \end{block}
    \vfill
    
    
\begin{block}{QuiPT concept}

In each permutation, for every observation, there are four possible results:

\begin{center}
\scalebox{0.85}{
$P(Target, Feature) = (1,1)) = p \cdot q$
}
\end{center}

\\

\begin{center}
\scalebox{0.85}{
$P(Target, Feature) = (1,0)) = p \cdot (1-q)$
}
\end{center}

\\

\begin{center}
\scalebox{0.85}{
$P(Target, Feature) = (0,1)) = (1-p) \cdot q$
}
\end{center}

\\

\begin{center}
\scalebox{0.85}{
$P(Target, Feature) = (0,0)) = (1-p) \cdot (1-q)$
}
\end{center}

\\

Where $p$ and $q$ are fractions of positive observations in target and
feature respectively. An another view at permutation test is therefore that we 
get a contingency table, which is to be tested for independence.
Computing probability of a such table with two constraints, $n_{1,\cdot} = n_{1,1} + n_{1,0}$ and
$n_{\cdot, 1} = n_{1,1} + n_{0,1}$, and 
conditioning on $n_{1,1}$, leads to hypergeometric distribution.
$n_{i,j}$ denotes number of observations for which 
\scalebox{0.85}{$(Target, Feature) = (i,j)$}

Thanks to this parametrization we replace a permutation test with the exact two-sided Fisher's test~\citep{lehmann1986testing}. 
%Information Gain is a way of deciding how strong is the evidence against independance.

\end{block}
\vfill 



}
\end{minipage}
\end{beamercolorbox}
\end{column}


%new column ------------------------------------------------------    

\begin{column}{.51\textwidth}
\begin{beamercolorbox}[center,wd=\textwidth]{postercolumn}
\begin{minipage}[T]{.95\textwidth}  
\parbox[t][\columnheight]{\textwidth}
{






%' \begin{block}{2-gram frequencies in sequences}
%' 
%' <<specSensPlot2, echo = FALSE, message=FALSE, fig.height=8, fig.width=14>>=
%' #ggplot(seq_inter_ct, aes(x = len, y = prop, fill = tar, label = paste0(round(prop, 4) * 100, "%"))) +
%' load("specsens.RData")
%' ggplot(filter(amylo_freq, n == 2), aes(x = variable, y = freq, fill = tar, colour = tar)) +
%'   geom_bar(stat = "identity", position = "dodge") + 
%'   scale_y_continuous("Frequency") +
%'   scale_x_discrete("Group ID\n") + 
%'   scale_fill_manual("Amyloid", values = c("no" = "skyblue", "yes" = "tan2")) +
%'   scale_colour_manual("Amyloid", values = c("no" = "skyblue", "yes" = "tan2")) +
%'   facet_wrap(~ enc, scales = "free_x", ncol = 1) +
%'   guides(colour = FALSE) +
%'   cool_theme
%' @
%' 
%' \end{block}
%' \vfill

\begin{block}{Signal peptides}
Secretory signal peptides:
        \begin{itemize}
          \item are short (20-30 residues) N-terminal amino acid sequences,
            \item direct a protein to the endomembrane system and next to the extracellular localization,
            \item possess three distinct domains with variable length and characteristic amino acid composition~\citep{hegde_surprising_2006}.
            \item are universal enough to direct properly proteins in different secretory systems; artifically introduced bacterial signal peptides can guide proteins in mammals~\citep{nagano_2014} and plants~\citep{moeller_2009},
            \item tag among others hormons, immune system proteins, structural proteins, and metabolic enzymes.
        \end{itemize}
\end{block}
\vfill

\begin{block}{Signal peptide prediction}
    During the test phase, each protein is fitted to two HSMMs representing respectively proteins with and without signal peptides. The probabilities of both fits and predicted cleavage site constitute the software output.    
    \begin{figure}
    \centering
    \resizebox{32.5cm}{!}{%
    \begin{tikzpicture}[->,>=stealth',shorten >=2pt,auto,node distance=9.5cm, thick]
      \tikzstyle{line} = [draw=black, color=blue!30!black!50, line width=4.5mm, -latex']
      \tikzstyle{main node} = [circle,fill=blue!20,draw, minimum size = 2.2cm, font=\itshape,
         align=center,  top color=white, bottom color=blue!50!black!70 ] %font=\sffamily\small\bfseries,
      %nodes
      \node[main node]          	(start') 	[]						{Start};	     
      \node[main node, bottom color=purple!70!black!70] 	(nregion') 	[right of=start',xshift=-5mm, yshift=15mm] 	{n-region};
      \node[main node, bottom color=pink!70!black!70] 	(hregion') 	[right of=nregion',xshift=-5mm,yshift=15mm] 	{h-region};
      \node[main node, bottom color=gray!70!black!70] 	(cregion') 	[right of=hregion',xshift=-5mm,yshift=-15mm] 	{c-region};
      \node[main node, bottom color=green!70!black!70] 	(mature') 	[right of=cregion',xshift=-5mm, yshift=-15mm] 	{Mature protein};
      
      %lines
      \path [line] (start')   edge node [left, color=black] {} (nregion');
      \path [line] (nregion') edge node [below, color=black] { } (hregion');
      \path [line] (hregion') edge node [below, color=black] { } (cregion');
      \path [line] (cregion') edge node [left, color=black] { } (mature');
      \draw [line] (start') to[out=340,in=200] (mature');
    \end{tikzpicture} }
    \end{figure}


    \end{block}
    \vfill   


\begin{block}{Summary and availability}
AmyloGram is a model-independent predictor of amylogenicity. Instead, it provides insight on the structural features present in the hot-spots. Moreover, AmyloGram recognises amylogenic sequences better than existing predictors.

\medskip

AmyloGram web-server: \url{smorfland.uni.wroc.pl/amylogram}.
\end{block}
\vfill


\begin{block}{Bibliography}
  \tiny{
  \bibliographystyle{apalike}
  \bibliography{amyloids}
  }
  \end{block}
  \vfill

}
\end{minipage}
\end{beamercolorbox}
\end{column}
\end{columns}  
\end{frame}
\end{document}
